%%%%%%%%%%%%%%%%%%%%%%%%%%%%%%%%%%%%%%%%%
% University Assignment Title Page 
% LaTeX Template
% Version 1.0 (27/12/12)
%
% This template has been downloaded from:
% http://www.LaTeXTemplates.com
%
% Original author:
% WikiBooks (http://en.wikibooks.org/wiki/LaTeX/Title_Creation)
%
% License:
% CC BY-NC-SA 3.0 (http://creativecommons.org/licenses/by-nc-sa/3.0/)
% 
% Instructions for using this template:
% This title page is capable of being compiled as is. This is not useful for 
% including it in another document. To do this, you have two options: 
%
% 1) Copy/paste everything between \begin{document} and \end{document} 
% starting at \begin{titlepage} and paste this into another LaTeX file where you 
% want your title page.
% OR
% 2) Remove everything outside the \begin{titlepage} and \end{titlepage} and 
% move this file to the same directory as the LaTeX file you wish to add it to. 
% Then add \input{./title_page_1.tex} to your LaTeX file where you want your
% title page.
%
%%%%%%%%%%%%%%%%%%%%%%%%%%%%%%%%%%%%%%%%%
%\title{Title page with logo}
%----------------------------------------------------------------------------------------
%	PACKAGES AND OTHER DOCUMENT CONFIGURATIONS
%----------------------------------------------------------------------------------------

\documentclass[12pt]{article}
\usepackage[english]{babel}
\usepackage[utf8x]{inputenc}
\usepackage{amsmath}
\usepackage{graphicx}
\usepackage[colorinlistoftodos]{todonotes}

\begin{document}

\begin{titlepage}

\newcommand{\HRule}{\rule{\linewidth}{0.5mm}} % Defines a new command for the horizontal lines, change thickness here

\center % Center everything on the page
 
%----------------------------------------------------------------------------------------
%	HEADING SECTIONS
%----------------------------------------------------------------------------------------

\textsc{\LARGE Imperial College London}\\[1.5cm] % Name of your university/college


%----------------------------------------------------------------------------------------
%	TITLE SECTION
%----------------------------------------------------------------------------------------

\HRule \\[0.4cm]
{ \huge \bfseries Smart Rods (Hardware) Interim Report}\\[0.4cm] % Title of your document
\HRule \\[1.5cm]
 
%----------------------------------------------------------------------------------------
%	AUTHOR SECTION
%----------------------------------------------------------------------------------------

\begin{minipage}{0.4\textwidth}
\begin{flushleft} \large
\emph{Author:}\\
Mattin \textsc{Mir-Tahmasebi} % Your name
CID: 00824754
\end{flushleft}
\end{minipage}
~
\begin{minipage}{0.4\textwidth}
\begin{flushright} \large
\emph{Supervisor:} \\
Prof. George \textsc{Constantinides} % Supervisor's Name
\end{flushright}
\end{minipage}\\[2cm]

% If you don't want a supervisor, uncomment the two lines below and remove the section above
%\Large \emph{Author:}\\
%John \textsc{Smith}\\[3cm] % Your name


%----------------------------------------------------------------------------------------
%	LOGO SECTION
%----------------------------------------------------------------------------------------

\includegraphics[width=0.5\textwidth]{ImperialCrest.jpg}\\[1cm] % Include a department/university logo - this will require the graphicx package
 
%----------------------------------------------------------------------------------------

\vfill % Fill the rest of the page with whitespace

\end{titlepage}


\begin{abstract}
Your abstract.
\end{abstract}

\section{Design Choices}

Several design choices have been made so far in the course of this project, each of which is justified below.

\subsection{Where should electronic complexity lie?}
\label{sec:complexity}

Some form of electronic device has to be used to identify the relative positions of the rods, but we have some choice in where this device exists. If we want to keep our product to be simply a set of rods, then the only place for the device is within the rods themselves. The rods would each need to be able to communicate with each other, to discern relative position, and also with either a controller that relays messages to the server or to the server directly.\\

Unfortunately, this solution brings several impracticalities. Firstly, the size of the rods has to be kept small, which constrains the size of battery we could embed into them. They will have to be capable of transmitting and receiving wireless messages, and even perhaps perform some amount of processing, so their power consumption will be too high for a battery of that size. Charging the batteries would also be impracticable; consider a classroom of around 30 students, each with a set of dozens of rods, and it becomes clear that the hassle of having to charge each device individually would outweigh any benefit of the product to a teacher.\\

Another issue concerns the identification of rods and their assignment to students: remember again that each student needs their own set of dozens of rods, and that any information these rods transmit will be attached to that child’s profile. The teacher would have to perform the frustrating and repetitive task of assigning each rod to a student, causing a considerable amount of inconvenience. Additionally, we must keep in mind that our target users are children, who may have the tendency to take rods from another student, misplace their own rods, or perhaps even throw rods around the classroom – all of these actions could cause sets of rods to be mixed together, leading to incorrect data being sent to the server, as progress from one student will be attached to the profile of another.\\

The last issue is economic: each of these rods will require several electronic components, including a processing unit and wireless transceivers, driving the cost of each unit up considerably. This is especially relevant as the smaller rods are easily lost in the classroom, so replacements would likely need to be purchased, increasing ongoing costs further.\\

An alternative solution to putting complexity in the rods is to design a playing board with which the rods can interact, and embed the majority of the electronics in that instead. The board would resemble a chess board, with grid squares guiding the placement of rods. The students arrange their rods on the board, which can identify the positions of the rods through some means, and communicate that data to the server. This alleviates all of the problems detailed above. The board is large and can house a larger battery than the rods, so there would be no power problem. There would only be one board per child so there are fewer devices to charge and much less work involved in assigning devices to children, also meaning that mixing of rods is not an issue, as it is the board which is unique to the child, not the rods. The cost to replace rods would be much lower as they would not contain as much electronics, and the overall cost would be lower as only the boards would require processing and communications equipment, as opposed to fitting one to every rod.\\ 

During an interview with a primary school teacher, we were informed that having a board may actually aid in the students’ learning, as having a grid could help them conceptualise how the rods fit together. 

\subsection{How should the board detect rod positions?}
\label{sec:detection}

Having settled on the use of a board, the next design choice to be made is what technology it will make use of to detect where rods have been placed. The chosen method needs to have the following characteristics to be viable: it must be able to \textit{reliably} detect rods and their positions, as any errors may mislead and confuse the child, weakening the product’s effectiveness as an educational tool. It should keep \textit{cost} low, as there could be up to hundreds of these products in use at an institution, and since they will likely be using public funding they will be under budget constraints. Lastly, it should remove as much \textit{complexity} from the rods as possible, for the reasons outlined in section \ref{sec:complexity}.\\

The descriptions of the different methods and related decision matrix follows.

\subsubsection{Weight}
\label{weight}

Rods could be given different weights, which are sensed by sensors on the board. Each weight would be mapped to a different length of rod.

\subsubsection{Magnetic Fields}
\label{magnets}

Magnets of varying strength could be inserted into rods to be detected by magnetometers in the board. Each strength would be mapped to a different length of rod.

\subsubsection{RFID}
\label{rfid}

Passive RFID chips could be inserted into the rods, and RFID sensors into the board. Each RFID chip would contain identifying information about the rod it is in.

\subsubsection{RGB Light}
\label{rgb}

\section{Some \LaTeX{} Examples}
\label{sec:examples}

\subsection{Sections}

Use section and subsection commands to organize your document. \LaTeX{} handles all the formatting and numbering automatically. Use ref and label commands for cross-references.

\subsection{Comments}

Comments can be added to the margins of the document using the \todo{Here's a comment in the margin!} todo command, as shown in the example on the right. You can also add inline comments  too:

\todo[inline, color=green!40]{This is an inline comment.}

\subsection{Tables and Figures}

Use the table and tabular commands for basic tables --- see Table~\ref{tab:widgets}, for example. You can upload a figure (JPEG, PNG or PDF) using the files menu. To include it in your document, use the includegraphics command as in the code for Figure~\ref{fig:frog} below.

% Commands to include a figure:
\begin{figure}
\centering
\includegraphics[width=0.5\textwidth]{frog.jpg}
\caption{\label{fig:frog}This is a figure caption.}
\end{figure}

\begin{table}
\centering
\begin{tabular}{l|r}
Item & Quantity \\\hline
Widgets & 42 \\
Gadgets & 13
\end{tabular}
\caption{\label{tab:widgets}An example table.}
\end{table}

\subsection{Mathematics}

\LaTeX{} is great at typesetting mathematics. Let $X_1, X_2, \ldots, X_n$ be a sequence of independent and identically distributed random variables with $\text{E}[X_i] = \mu$ and $\text{Var}[X_i] = \sigma^2 < \infty$, and let
$$S_n = \frac{X_1 + X_2 + \cdots + X_n}{n}
      = \frac{1}{n}\sum_{i}^{n} X_i$$
denote their mean. Then as $n$ approaches infinity, the random variables $\sqrt{n}(S_n - \mu)$ converge in distribution to a normal $\mathcal{N}(0, \sigma^2)$.

\subsection{Lists}

You can make lists with automatic numbering \dots

\begin{enumerate}
\item Like this,
\item and like this.
\end{enumerate}
\dots or bullet points \dots
\begin{itemize}
\item Like this,
\item and like this.
\end{itemize}

We hope you find write\LaTeX\ useful, and please let us know if you have any feedback using the help menu above.

\end{document}