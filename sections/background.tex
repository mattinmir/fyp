\chapter{Background}

\section{Cuisenaire\textsuperscript{\textregistered} Philosophy}
Background research began with discovering how Cuisenaire\textsuperscript{\textregistered} rods are used and why they are useful. John V. Trivett, in his article, \emph{The Cuisenaire Rods—Numbers in Colour}, from the Journal of the Association of Teachers of Mathematics \cite{johnv.trivett1962} talks about how abstracting the notation from the teaching of mathematics allows students to form conceptual instincts about ratios between values. He argues this is more valuable than committing equations and expressions to memory to be regurgitated during examinations. This idea is reinforced by Tony Wing in his article in the same journal, \emph{Working Towards Mental Arithmetic... And (Still) Counting}\cite{tonywing1996}, where he states that it is important for children to learn to play with the rods \say{before conventional number names and symbols are attached to them}.\\

This is something to keep in mind throughout the design process: we want to adhere to that principle of abstraction and not construct the rods in a way that would cause the children to treat them as numbers, for example, by having each rod's length written on it. \\

\section{Teacher Advice}

Further research was carried out by contacting primary school maths teachers who have made use of Cuisenaire\textsuperscript{\textregistered} rods, to ask them about the benefits of the rods and what could be improved. A criticism of the rods is that teachers need to record proof of students' work for administrative purposes, and the rods do not lend themselves to this very well because the children are required to draw diagrams of their solutions in workbooks, which takes considerable time from the exercise. This is a problem that our product will likely be able to solve, as an electronic system will be able to record every attempt a student makes at an answer, meaning that our records will be more detailed than what the children could draw themselves.\\

Another complaint was that, often, neither students nor teachers know how to properly make use of the rods when presented with them. This creates, for us, a task in human-computer interaction, as we will have to design a system that is intuitive to use with little mental effort required by the user. This includes being intuitive to set up (for teachers) and intuitive to operate (for children).\\

It was warned that students may not treat the product with care and so it should be able to withstand bending or general mistreatment at the hands of the children. This will inform our choice of material and construction method once we begin production. \\


\section{Similar Products}
Online searching yielded very few internet-enabled educational devices for the KS1 demographic, excluding educational applications for tablets and computers. This means that we will not be able to compare the hardware aspect of our product to existing designs, and will be setting precedents with some of our design choices. While this gives us the opportunity to explore and discover new methods of integrating technology into teaching, it does mean that the chance of our ideas being unsuccessful are increased, as we do not have a foundation of work to build upon. This effect can be mitigated via consultation of our contacts in schools and by working closely with our supervisor who can help guide our path.




