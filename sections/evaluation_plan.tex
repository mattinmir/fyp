\chapter{Evaluation Plan}

The majority of the evaluation for this project is qualitative and subjective; the true judge of the success of the product will be the reactions of the teachers and students who will use the product. We will want to know if our design met our aim of helping teachers to keep a record of students' work, and to more easily track and monitor student progress, and also if it provides a friendly user experience for both teachers and students. \\

This should be achieved through communication with primary maths teachers, and visits to nearby schools where we can see how a classroom might make use of the product. We should observe the way in which the product is used, making note of unexpected ways in which the product is used or handled. Not being part of the target demographic for the product, it is likely there are some use-cases which we have not anticipated, and such instances would be taken into account if design for the product were to be continued further. The students should be monitored to see what, if any, frustrations they have with operating the product, and if they enjoyed using it. The teachers should be surveyed about the ease of set-up of the product (including charging and distribution among children) and about the merit the product has as an educational tool compared to other methods they have used. \\

The ideal evaluation would be a long-term study of a class using the product over the course of a year, along with a similar class using traditional rods or other methods of teaching, and comparing each class' performance and satisfaction over that year. This is, unfortunately, out of the scope of this project, so we have to settle with a much shorter-term survey. \\

Since this report focusses only on the hardware aspect of the product, evaluation considerations must focus on that aspect too. While it can be difficult to separate the concepts of software and hardware when trying to evaluate the product, one feature that is exclusive to the hardware is the reliability of rod placement detection. The accurate detection of rods is vital in ensuring children do not get confused when using the product, not only because it may cause them to misunderstand the lesson they are supposed to learn, but if they suspect the product does not function properly, they may be less inclined to engage with it in future. \\

To evaluate reliability, multiple possible configurations of rods should be tested and the response of the board should be recorded. The test should include extreme and edge cases, such as filling the entire board with rods, quickly placing and removing a rod, and adding many rods in quick succession. There is no margin of error, so every test needs to pass.

























